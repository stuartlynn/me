%% start of file `template.tex'.
%% Copyright 2006-2013 Xavier Danaux (xdanaux@gmail.com).
%
% This work may be distributed and/or modified under the
% conditions of the LaTeX Project Public License version 1.3c,
% available at http://www.latex-project.org/lppl/.


\documentclass[11pt,a4paper,sans]{moderncv}        % possible options include font size ('10pt', '11pt' and '12pt'), paper size ('a4paper', 'letterpaper', 'a5paper', 'legalpaper', 'executivepaper' and 'landscape') and font family ('sans' and 'roman')

% moderncv themes
\moderncvstyle{casual}                             % style options are 'casual' (default), 'classic', 'oldstyle' and 'banking'
\moderncvcolor{blue}                               % color options 'blue' (default), 'orange', 'green', 'red', 'purple', 'grey' and 'black'
%\renewcommand{\familydefault}{\sfdefault}         % to set the default font; use '\sfdefault' for the default sans serif font, '\rmdefault' for the default roman one, or any tex font name
%\nopagenumbers{}                                  % uncomment to suppress automatic page numbering for CVs longer than one page

% character encoding
\usepackage[utf8]{inputenc}                       % if you are not using xelatex ou lualatex, replace by the encoding you are using
%\usepackage{CJKutf8}                              % if you need to use CJK to typeset your resume in Chinese, Japanese or Korean

% adjust the page margins
\usepackage[scale=0.75]{geometry}
%\setlength{\hintscolumnwidth}{3cm}                % if you want to change the width of the column with the dates
%\setlength{\makecvtitlenamewidth}{10cm}           % for the 'classic' style, if you want to force the width allocated to your name and avoid line breaks. be careful though, the length is normally calculated to avoid any overlap with your personal info; use this at your own typographical risks...

% personal data
\name{Stuart}{Lynn}
\title{Resumé title}                               % optional, remove / comment the line if not wanted
\address{1817 South Allport}{60608 Chicago}{USA}% optional, remove / comment the line if not wanted; the "postcode city" and and "country" arguments can be omitted or provided empty
\phone[mobile]{+1~(312)~394~0797}                   % optional, remove / comment the line if not wanted
\email{stuart@zooniverse.org}                               % optional, remove / comment the line if not wanted


% to show numerical labels in the bibliography (default is to show no labels); only useful if you make citations in your resume
%\makeatletter
%\renewcommand*{\bibliographyitemlabel}{\@biblabel{\arabic{enumiv}}}
%\makeatother
%\renewcommand*{\bibliographyitemlabel}{[\arabic{enumiv}]}% CONSIDER REPLACING THE ABOVE BY THIS

% bibliography with mutiple entries
%\usepackage{multibib}
%\newcites{book,misc}{{Books},{Others}}
%----------------------------------------------------------------------------------
%            content
%----------------------------------------------------------------------------------
\begin{document}
%-----       letter       ---------------------------------------------------------
% recipient data

\date{January 12, 2015}
\opening{Dear Sir or Madam,}
\closing{Yours faithfully,}
\makelettertitle

Thank you for considering me for the position of Education \& Public Outreach Project Manager for the LSST. Having watched the LSST take shape these past few years I would be thrilled to have a major role communicating and sharing the excitement felt by the scientific field for this project, the discoveries it will make and finding new and novel ways to invite the public to take part in this unique endeavor. 

Over the past 5 years I have worked in various different roles within
the the Zooniverse and currently act as the Technical Lead of the project. During that time I have developed a strong appreciation for the power of giving the public access to real data.. The authenticity of the experiences we built have been essential to not only establishing citizen science as a legitimate research method but in creating a powerful platform to engaging members of the public in science in a transformative way.

To me LSST represents a golden opportunity to explore this approach of data driven engagement even further. One of the things that drew me to apply for this job is the policy the LSST has taken to provide a pipeline of data dedicated to EPO activities. Allowing citizen scientists, students, planetarium visitors and the public in general to access images and data from the telescope as it becomes
 available to the scientific community is a huge opportunity to evolve how we think of public participation in science. The challenges to doing so are the same challenges we faced with the Zooniverse project: developing flexible powerful intuitive tools and interfaces that can build and facilitate meaningful experiences with this data at its core. 
  
Not only does data driven public participation provide deep engagement opportunities but it also represents an opportunity to scale this participation to large numbers of people. The ubiquity of the Internet allowed us with Zooniverse to reach over 1 million people,LSST has the potential to reach even larger numbers of individuals. I believe I am uniquely qualified to address the significant technical
and creative challenges of operating at this scale.

Like with any effort to engage the public in science, assessment is key to success. Not just for providing evidence of impact but also to improve the approaches and tools being developed. In my role at Zooniverse I worked with researchers at Syracuse University to use traditional surveys but also data mining techniques to help understand users interaction with our projects. We found ways to assess not only user engagement but, more interestingly,  behavior and motivation. One of these efforts exposed conceptual problems with our tutorial system and another showed that participants who took part in the Snapshot Serengeti project learned how to identify previously unknown species as they participated in the project. As a result I have become very interested in using both traditional methods and exploring new ways to use data to create assessment metrics for EPO.

I would love to have the opportunity to bring the lessons I have learned and the skills I have acquired from participating in and running both traditional outreach programs and novel online projects to LSST. I believe that LSST has the opportunity to push EPO into new territories where students and the public are not just spectators but are invited to become part of the discovery process. I can think of few opportunities that are more exiting. 



\end{document}


%% end of file `template.tex'.
